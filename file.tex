\documentclass{article}
\usepackage[utf8]{inputenc}
\usepackage{graphicx}

\title{Movie Review}
%\author{dushyantk1509 }
\date{April 2017}

\begin{document}

\maketitle

\section{3 Idiots}
\section{review by dushyant}
OAKLAND -- "Life is a race: If you're not fast enough, you'll get trampled," booms engineering college dean Viru Sahastrabudhhe (Boman Irani). The stakes are high: in India 2.0, students whose parents have sacrificed everything for their education are expected to reach the top. But three brainy misfits -- Farhan (R. Madhavan), Raju (Sharman Joshi) and Rancho (Aamir Khan) -- find a common bond when they realize that their futures aren't exactly inscribed in a mathematics book.

As Rancho, a young engineer of dazzling inventiveness, Khan conveys smarts, mischief and, finally, compassion, as he spurs his friends on to greatness before he mysteriously disappears.

\section{Black}
\begin{figure}[h!]
  \includegraphics[width=25mm,scale=0.5]{black.jpg}
\end{figure}
\begin{flushleft}
\textsf{\textbf{Director}: Sanjay Leela Bhansali\\ 
        \textbf{Music}: Monty\\
        \textbf{Starring} : Amitabh Bachchan, Rani Mukherjee, Ayesha Kapoor, Shernaz Patel}\\
\end{flushleft}  
                           
         \vspace{2mm}
Black by Sanjay Leela Bhansali is a tribute to the famous blind and deaf laureate, Helen Keller. The story is the directors interpretation of Helen's remarkable life and the role her teacher Annie Sullivan played in her life.
\par
The story begins as the protagonist Michelle McNally (Rani Mukerji), a blind and deaf woman born in an Anglo Indian family reflects on her life. She is in search of her lost teacher Debraj Sahai( Amitabh Bachhan) who made her what she is. She finds him after 20 long years, near her home with his memory completely gone that he cannot even recognize her.

\end{document}

